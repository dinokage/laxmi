\documentclass[a4paper]{jsarticle}
\pagestyle{empty} 
\usepackage{amssymb} 
\usepackage{otf} 
\usepackage{bm} 
\usepackage{amsmath} 
\newtheorem{Thm}{定理}%[section]
\newtheorem{Def}{定義}%[section]
\newtheorem{Lem}{補題}%[section]
\newtheorem{Rem}{注意}%[section]
\begin{document}
\begin{center}
\LARGE \textbf{\underline{Paillier暗号}}
\end{center}
\begin{enumerate}
\item[\textbf{(1)}] \textbf{準備}\\
Paillier暗号は,次のような性質を利用している.$n \in \mathbb{N}^+$のとき,二項定理より,
\begin{align*}
(1 + x)^n = \sum^{n}_{j = 0}{}_n\mathrm{C}_{j}x^j\equiv 1 + nx.\ \ \  (\mathrm{mod}\  n^2)\tag*{(1)}
\end{align*}
そこで,$y := (1 + x)^n $とすると,$x = (y - 1)/n\ (\mathrm{mod}\  n^2)$である.ここで,関数$L(x)$を次のように定義する.
\begin{align*}
\label{funcL}
L(x) := \dfrac{x - 1}{n}.\tag*{(2)}
\end{align*}
この関数は,後でPaillier暗号の復号の際に使う.
\begin{align*}
L(y\ (\mathrm{mod}\  n^2)) =\dfrac{y\ (\mathrm{mod}\  n^2) - 1}{n}\ \ \ (\mathrm{mod}\  n)\equiv\dfrac{(1 + nx)- 1}{n}=x.\tag*{(3)}
\end{align*}

\item[\textbf{(2)}]\textbf{鍵生成}
\begin{itemize}
\item $\gcd(pq,(p-1)(q-1)) = 1$を満たす大きな素数$p,q$を選ぶ.
\item 公開鍵$\bm{pk}$は,$n\leftarrow pq, \ g\leftarrow (1+\alpha n)\beta^n$である.ただし,$\forall\alpha,\forall\beta\in\mathbb{Z}^{*}_{n^2}$である.
\item 秘密鍵$\bm{sk}$は,$\lambda\leftarrow \mathrm{lcm}(p-1, q-1)$である.
\end{itemize}




\item[\textbf{(3)}]\textbf{暗号化アルゴリズム}\\
公開鍵$\bm{pk}$と$\gcd(r,n) = 1$を満たす乱数$r\in\mathbb{Z}^{*}_{n^2}$を用いて,次のようにメッセージ$m\in\mathbb{Z}_{n}$を暗号化する.つまり,$c$を暗号文,$E$を暗号化関数とすると,
\begin{align*}
c=E(m,r):=g^mr^n.\ \ \ (\mathrm{mod}\  n^2)\tag*{(4)}
\end{align*}



\item[\textbf{(4)}]\textbf{復号化アルゴリズム}\\
秘密鍵$\bm{sk}$と\ref{funcL}の関数$L$を用いて復号する.
\begin{align*}
\label{decp}
\dfrac{L(c^\lambda\ (\mathrm{mod}\  n^2))}{L(g^\lambda\ (\mathrm{mod}\  n^2))}  = m.\ \ \ (\mathrm{mod}\  n)\tag*{(5)}
\end{align*}
\ref{decp}を計算するために,まず$r^{\lambda n} \equiv 1 \ (\mathrm{mod}\  n^2)$を示す必要がある.
%%%%%%%%%%%%%%%%%%%%%%%%%%%%%%%%%%%%%%%%%%%%%%%%%
\Lem \label{Thm1}$\gcd(r,n) = 1$を満たす$r\in\mathbb{Z}^{*}_{n^2}$について$r^{\lambda n} \equiv 1 \ (\mathrm{mod}\  n^2)$が成り立つ.\\
%%%%%%%%%%%%%%%%%%%%%%%%%%%%%%%%%%%%%%%%%%%%%%%%%
\textit{proof.\ }Eulerの整関数より$\varphi(n^2) = n(p-1)(q-1)$であり,$\gcd(pq,(p-1)(q-1)) = 1$なので,原始根定理より$\mathbb{Z}^{*}_{n^2}$には,位数$n$と位数$(p-1)(q-1)$の部分群が存在する.したがって,$\gcd(r,n) = 1$を満たす$r\in\mathbb{Z}^{*}_{n^2}$について, Eulerの定理の精密化\footnote{Refined Euler's Theorem}より,
\begin{align*}
r^\lambda\equiv 1.\ \ \ (\mathrm{mod}\  n)\tag*{(6)}
\end{align*}
つまり,ある整数$Q$を用いて$r^\lambda:=1 + nQ$と表すことができる.よって,
\begin{align*}
r^{\lambda n} = (1 + nQ)^n = 1 + n^2 (Q +\cdots) \equiv 1.\ \ \ (\mathrm{mod}\  n^2)\tag*{(7)}
\end{align*}
以上より,補題\ref{Thm1}は示された.$\blacksquare$


%%%%%%%%%%%%%%%%%%%%%%%%%%%%%%%%%%%%%%%%%%%%%%%%%
補題\ref{Thm1}を用いて,$L(c^\lambda\ (\mathrm{mod}\  n^2))$を具体的に計算すると,$\gcd(\alpha, n) = 1, \gcd(\beta, n) = 1$なので,
\begin{align*}
\label{LC}
L(c^\lambda\ (\mathrm{mod}\  n^2)) &= L(g^{m\lambda}\cdot r^{\lambda n}\ (\mathrm{mod}\  n^2))=L(g^{m\lambda}\ (\mathrm{mod}\  n^2))&\\
&= \dfrac{(1+\alpha n)^{m\lambda}\beta^{nm\lambda}\ (\mathrm{mod}\  n^2)-1}{n}&\\
&\equiv\dfrac{(1 + \alpha nm\lambda)-1}{n}=\alpha m\lambda.\ (\mathrm{mod}\  n)\tag*{(8)}&
\end{align*}
以上\ref{decp},\ref{LC}より,
\begin{align*}
\dfrac{L(c^\lambda\ (\mathrm{mod}\  n^2))}{L(g^\lambda\ (\mathrm{mod}\  n^2))} = \alpha m\lambda\cdot(\alpha\lambda)^{-1} = m.\ (\mathrm{mod}\  n)\tag*{(9)}
\end{align*}

\item[\textbf{(5)}]\textbf{暗号化関数の準同型性}\\
Paillier暗号化関数$E$は準同型写像である.乱数$r_1, r_2\in\mathbb{Z}^{*}_n$とメッセージ$m_1, m_2\in\mathbb{Z}_{n}$を暗号化した暗号文$E(m_1,r_1),E(m_2,r_2)$の積を考える.
\begin{align*}
\label{kahou1}
E(m_1,r_1)\times E(m_2,r_2)&=g^{m_1}r_1^n\times g^{m_2}r_2^n\\
&=g^{m_1+m_2}(r_1r_2)^n&\\
&=E(m_1+m_2, r_1r_2).\ \ \ (\mathrm{mod}\  n^2)&\tag*{(10)}
\end{align*}
したがって,メッセージ$m\in\mathbb{Z}_{n}$のスカラー倍は次のように表すことができる.$N\in\mathbb{Z}_n$において,
\begin{align*}
\label{kahou2}
\prod_{i = 1}^{N}E(m,r_i)&=E(m,r_1)\times E(m,r_2)\times \cdots \times E(m,r_N)&\\ 
&=(g^{m})^N\times \prod_{i=1}^{N} r_i=E(N\times m, r_1r_2\cdots r_N).\ \ (\mathrm{mod}\  n^2)&\tag*{(11)}
\end{align*}
\ref{kahou1},\ref{kahou2}のように暗号化したまま平文の和やスカラー倍を求められる性質を\textbf{加法準同型性}と呼ぶ.
\end{enumerate}
\end{document}
